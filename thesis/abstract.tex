\begin{abstract}
	This thesis provides a comprehensive exploration, implementation, and thorough experimental evaluation of algorithms for polyline simplification under the global Fréchet distance. To our knowledge, this work presents the first extensive experimental study in this specific context. 

	We detail all necessary subproblems for these algorithms and discuss optimizations that lead to practically feasible runtimes.
	A key conceptual contribution is the classification of approaches to Fréchet-related problems into three categories: explicit, implicit, and semi-explicit. While we show that prior art falls into the explicit category, we introduce novel implicit and semi-explicit variants. These variants are based on decision problems that abstract direct value comparisons, enabling, for example, the solution of many problems in the Euclidean metric without square root computations. 

	A main algorithmic contribution is a new method that computes, for a given polyline, all its possible simplifications across all values of \(\varepsilon\), including the intervals of \(\varepsilon\) for which each simplifications is valid. This algorithm can be implemented in \(\hat{\mathcal{O}}(n^4\log n)\) time and \(\hat{\mathcal{O}}(n^3)\) space for a polyline of length \(n\). We utilize this to analyze the relationship between \(\varepsilon\) and simplification size. 

	Furthermore, We demonstrate how a global Fréchet distance variant can be derived from the classical Imai-Iri algorithm. The construction yields an algorithm equivalent to the one by \citeauthor{global_curve_simplification}, yet offers a more intuitive understanding of its underlying principles. 
\end{abstract}
