\section{Simplification Algorithm from van Kreveld et al.}
\label{sec:algorithm_implementation}

In this section we will sketch \citeauthor{on_optimal_polyline_simplification_using_the_hausdorff_and_frechet_distance}'s algorithm for polyline simplification as well as a simplified version of \citeauthor{computing_the_frechet_distance_between_two_polygonal_curves}'s algorithm to decide if two polylines have Fréchet distance of at most \(\varepsilon\). We will see an example of these algorithms and end with optimizations that can be applied.

\subsection{Fréchet Distance Decision Algorithm}
\label{ssec:alt_godau}
The simplification algorithm we will describe heavily uses a subroutine to solve the following problem: Given \(\varepsilon > 0\), a polyline \(P\) of length \(p\), a subpolyline \(P[j' + t' \dots j+1]\), and a line segment \(\overline{P(i')P(i)}\) (where \(i' < i \leq p, j' \leq j < p\in \N\), and \(t' \in [0, 1]\)), decide whether there exists \(t \in [0, 1]\) such that \(\delta^F(P[j' + t' \dots j + t], \overline{P(i')P(i)}) \leq \varepsilon\). If so, return the smallest such \(t\).  

This can be solved by a simplified version of \citeauthor{computing_the_frechet_distance_between_two_polygonal_curves}'s algorithm to decide if the Fréchet distance of two polylines is at most \(\varepsilon\). We present the algorithm in the specific form needed for this problem. 

By definition of the Fréchet distance, we require \(\delta(P(j' + t'), P(i')) \leq \varepsilon\), i.e., the starting points must be close. If this fails, we immediately return ``no solution". We set \(t_0 \coloneq \hat t_0'(\overline{P(j)P(j+1)}, P(i))\) and \(t_1 \coloneq \hat t_1'(\overline{P(j)P(j+1)}, P(i))\). Similarly, as the end points must match, we get that \(t \in [t_0, t_1]\). Thus if there is no such interval we can return that there is no solution.

We distinguish the two cases \(j' = j\) and \(j' < j\). 
\begin{enumerate}
	\item[\(j' = j\): ] The subpolyline reduces to a single line segment \(\overline{P(j)P(j+1)}\). The constraints simplify to \(t \in [t', 1] \cap [t_0, t_1]\), which is feasible if and only if \(t' \leq t_1\). The solution is then \(\max(t', t_0)\).

	\item[\(j' < j\): ] We iterate over the intermediate points \(P(k) = P(j'+1), \dots, P(j)\) and compute the solutions \(\hat t_0'(\overline{P(i')P(i)}, P(k))\), and \(\hat t_0'(\overline{P(i')P(i)}, P(k))\). We maintain the first reachable point \(s\) on the line segment \(\overline{P(i')P(i)}\) (initially \(s = 0\)) for these points and test if \(s \leq \hat t_1'(\overline{P(i')P(i)}, P(k))\). If this fails we return that there is no solution. Otherwise we update \(s = \max(s, \hat t_0'(\overline{P(i')P(i)}, P(k)))\). Finally, we have fully traversed the line segment and most of the subpolyline. At this point the solution is merely \(t_0\).
\end{enumerate}

The correctness for the first case follows directly from \cref{lem:distance_properties}. The correctness of case \(j' < j\) can also be shown relatively easily. 

\begin{lemma}\label{lem:ag-neq}
	Let \(P = \angl{u_0, \dots, u_p}\) be a polyline of length \(p\) and \(e = \overline{v_0v_1}\) be a line segment. Then 
	\(\delta^F(P, e) \leq \varepsilon\) if and only if 
	\begin{enumerate}
		\item \(\delta(u_0, v_0) \leq \varepsilon\),
		\item \(\delta(u_p, v_1) \leq \varepsilon\), and
		\item There are \(x_i \in [\hat t_0'(e, u_i), \hat t_1'(e, u_i)]\) for \(i \in \set{1, \dots, p - 1}\) with \(x_1 \leq \cdots \leq x_{p-1}\).
	\end{enumerate}
\end{lemma}
\begin{proof}
	For the forward direction there are function \(f \in \mathcal{C}([0,1], [0, p])\) and \(g \in \mathcal{C}([0,1], [0, 1])\) with \(\delta(P(f(t)), e(g(t))) \leq varepsilon\) for all \(t \in [0, 1]\). By plugging in \(0\) or \(1\) we get the first two properties. For the third one pick \(x' \in [0, 1]\) such that \(f(x) = i\) and pick \(x_i = g(x)\). Such an \(x\) exists as \(f\) is continuous (but it may not be unique). Then \(\delta(P(i), e(x_i)) \leq \varepsilon \) implying that \(x_i \in [\hat t_0'(e, u_i), \hat t_1'(e, u_i)]\). Furthermore, \(x_1 \leq \cdots \leq x_{p-1}\) as \(1 < \cdots < p-1\) and both \(f\) and \(g\) are monotonous.

	For the backward direction, we must construct suitable function \(f\) and \(g\). For \(g\) we pick the identity and we define \(f(x_i) = i\) for all \(i \in \set{1, \dots, p - 1}\) and additionally \(f(0) = 0\) and \(f(1) = p\). All other points are linearly interpolated between the defined points resulting in \(f((1-t)x_i + tx_{i+1}) = i + t\) for \(t\in [0,1]\) (where we extend \(x_0 = 0\) and \(x_p = 1\)). The correctness of \(f\) follows directly from the convexity property from \cref{lem:distance_properties} as well as \(x_1 \leq \cdots \leq x_{p-1}\) which guarantees monotonicity.
\end{proof}

\subsection{Polyline Simplification Algorithm}
\label{ssec:simple_algo_main}

Here we outline the global polyline simplification algorithm from \citeauthor{on_optimal_polyline_simplification_using_the_hausdorff_and_frechet_distance} which we have implemented and tested for the Euclidean distance, the Manhattan distance, and the Chebyshev distance. 

Similar to the decision problem we use a dynamic program in which we store the earliest reachable points although in a more complicated manner. We build a 3D table \(DP[k,i,j] \in [0, 1] \cup \set{\infty}\) for each triple \((k, i, j)\) with \(k, i \in \set{0, \dots, n}\) and \(j \in \set{0, \dots, n - 1}\) where \(n\) is the length of the given polyline \(P\). 

Each entry \(DP[k, i, j]\) stores the smallest \(t \in [0, 1]\) s.t. there is a simplification \(Q\) of \(P[0 \dots i]\) with exactly \(k\) line segments with \(\delta^F(Q, P[0\dots j + t]) \leq \varepsilon\). If no such \(t\) exists we store \(\infty\) in that entry. 
To retrieve the simplification from this table, we find the smallest \(k^*\) s.t. \(DP[k^*, n, n - 1]\) exists, i.e., there is a simplification \(Q\) of the whole polyline with \(k^*\) many line segments that has \(\delta^F(Q, P[0\dots n - 1 + t]) \leq \varepsilon\) for some \(t \in [0, 1]\) so we can complete the simplification by simply going from \(n-1+t\) to \(n\) on the last line segment of \(P\) while staying on the last point of the simplification. 

For \(k = 0\) it is trivial to compute the entries. If \(i > 0\) no simplification exists, i.e., we can store \(\infty\), as it is impossible to create a simplification of size \(0\) that goes to any point other than the initial point. To find \(DP[0, 0, j]\) we only need to compute the distances from \(P(0)\) to the points \(P(j)\) (see \cref{fig:simpl_init}). Until the first \(j\) with \(\delta(P(0), P(j)) > \varepsilon\) we can store \(0\). From the first such \(j\) we store \(\infty\). Since the simplification consists only of a single point \(P(0)\), the condition \(\delta^F(Q, P[0\dots j + t]) \leq \varepsilon\) simplifies to \(\delta^F(P[0 \dots 0], P[0 \dots j + t])\) and as we cannot move on the single point \(P(0)\) the earliest reachable point on each line segment must be \(t = 0\). 

The correctness of this initialization can be shown rather easily. 
\begin{lemma}
  Let \(P = \angl{P(0)}\) a polyline consisting of a single point and \(Q\) be a polyline of size \(n\). Then \(\delta^F(P, Q) \leq \varepsilon\) if and only if \(\delta(P(0), Q(i)) \leq \varepsilon\) for all \(i \in \set{0, \dots, n}\). 
\end{lemma}
\begin{proof}
  For the forward direction, there exist functions \(f\) and \(g\) such that \(\delta(P(f(t)), Q(g(t))) \leq \varepsilon\) for all \(t\in [0,1]\), where \(f(0) = g(0) = 0\), \(f(1) = 0\), \(g(1) = n\), and \(f\) and \(g\) are  monotone. This implies that \(\delta(P(0), Q(g(t))) \leq \varepsilon\). As \(g\) is continuous and \(g(0) = 0 \leq g(1) = n\) there must be some \(t \in [0,1]\) with \(g(t) = x\) for any \(x\in \set{0, \dots, n}\).
  
  For the backward direction, we show that every point on \(Q\) has distance at most \(\varepsilon\) from \(P(0)\) thus we can choose any suitable function \(g\) (for \(f:[0,0] \to [0,0]\) there is only one possibility). This follows directly from the convexity property from \cref{lem:distance_properties} applied to all individual line segment on \(Q\).
\end{proof}

\begin{figure}[b]
  \centering
  \includegraphics{tikz-fig/simpl_init.pdf}
  \caption{Initialization of the simplification algorithm for \(k = 0, i = 0\). Only the points in the circle until \(P(2)\) are reachable.}
  \label{fig:simpl_init}
\end{figure}


As for the other entries, the authors show that the entry \(DP[k, i, j]\) can be computed by the minimization over all \(i' < i\) and \(j' \leq j\). We lookup the value \(t' \coloneq DP[k-1, i', j']\) and test if there is a \(t \in [0, 1]\) to which we can proceed, i.e., \(\delta^F(P[j' + t' \dots j + t], \overline{P(i')P(i)}) \leq \varepsilon\) and return the smallest such \(t\) if possible. This can be done with the algorithm from \citeauthor{computing_the_frechet_distance_between_two_polygonal_curves} with the mentioned modifications. Among all those candidates \(t\) we choose the minimal one. If no such \(t\) exists we can store \(\infty\).

\begin{algorithm}[ht]
  \DontPrintSemicolon
  \KwData{Polyline \(P\) of length \(n\), \(\varepsilon > 0\)}
  \KwResult{Smallest \(\varepsilon\)-simplification of \(P\)}
  \BlankLine
  \(DP \gets Array((n + 1, n + 1, n))\) initialized with \(\infty\) \;
  \For{\(j = 0, \dots, n\)}{
    \If{\(\delta(P(0), P(j)) > \varepsilon\)}{
      \Break
    }
    \(DP[0, 0, j] \gets 0\)
  }
  \For{\(k=1,\dots\) until \(DP[k, n, n-1] \neq \infty\)}{
    \For{\(i=0,\dots, n\)}{
      \For{\(j=0,\dots, n-1\)}{
        \For{\(i' < i\)}{
          \For{\(j' \leq j\)}{
            Let \(t' \gets DP[k-1, i', j']\)\;
						\If{\(t' = \infty \)}{
							\Continue
						}
            Let \(t \gets AltGodau(P[j' + t' \dots j + 1], \overline{P(i')P(i)}, \varepsilon)\)\;
            \(DP[k, i, j] \gets \min(DP[k, i, j], t)\)
          }
        }
      }
    }
  }
  \caption{PolylineSimplification(\(P, \varepsilon\))}
  \label{algo:simplify_simple}
\end{algorithm}

There are \(\O(k^* n^2)\) many iterations to fill in the table and each entry requires \(\O(n^2)\) many computations to find the minimum where \(k^*\) is the size of the output simplification. Each call to the \(AltGodau\) subroutine requires linear runtime thus in total we have \(\Oh(k^*n^5)\) run time.

\subsection{Examples}
Before discussing possible optimizations, we want to go through an example for the algorithm as well as the \citeauthor{computing_the_frechet_distance_between_two_polygonal_curves} subroutine. This gives us more intuition for the algorithms as well as ideas for further optimizations. Because of the cubic number of entries it is unreasonable to show all computations thus we only show the computations that lead to the simplification and the subroutine for them. 

\begin{figure}
  \centering
  \includegraphics{tikz-fig/poly-ex-main.pdf}
  \caption{Example polyline with circles with radius \(\varepsilon\) drawn around all points. We note the following relations which may be hard to notice: \(\delta(P(2), P(4)) = \delta(P(2), P(5)) = \varepsilon\), \(\delta(P(2), P(3)) > \varepsilon\), \(\delta(P(2), P(6)) > \varepsilon\).}
  \label{fig:poly-ex-main}
\end{figure}

For the polyline and \(\varepsilon\) in \cref{fig:poly-ex-main} we initialize the table layer for \(k = 0\) by only setting the value \(DP[0,0,0]\) to \(0\) as \(\delta(P(0), P(1)) > \varepsilon\). Even though \(\delta(P(0), P(4)) \leq \varepsilon\) the value \(DP[0,0,4]\) is \(\infty\) as there are points between that already fail.

For the layer \(k = 1\) we can only proceed from the entry \(DP[0,0,0] = 0\) as it is the only one we found on the previous layer. All triples \(k, i, j\) for which a valid entry in \([0, 1]\) can be found for the layer \(k = 1\) are listed in \cref{tab:exlayer1}. We do not list the respective \(i', j'\) and \(t'\) as there is only one possible entry.
\begin{table}[ht]
\centering
\begin{tabular}{|ccc|}
\hline
$(1,1,0)$ & $(1,1,1)$ & $(1,1,2)$ \\
$(1,2,0)$ & $(1,2,1)$ & $(1,2,2)$ \\
$(1,2,3)$ & $(1,2,4)$ & $(1,2,5)$ \\
$(1,3,2)$ & $(1,3,3)$ & $(1,4,0)$ \\
$(1,4,1)$ & $(1,4,2)$ & $(1,5,0)$ \\
$(1,5,1)$ & $(1,5,2)$ & \\
\hline
\end{tabular}
\caption{Valid entries for layer \(k = 1\). All proceed from \((0,0,0)\).}
\label{tab:exlayer1}
\end{table}

Let us explicitly go through the AltGodau subroutine for the entry \((1, 2, 3)\), i.e., A simplification consisting of one line segment for the subpolyline \(P[0\dots 2]\) that has distance of at most \(\varepsilon\) from \(P[j' + t'\dots j + t] = P[0 \dots 3 + t]\) where \(t\) is gotten from the subroutine. We consider the line segment \(\overline{P(i')P(i)} = \overline{P(0)P(2)}\) and the polyline segment \(P[j' + t' \dots j + 1] = P[0 \dots 4]\) and perform the algorithm. 

\begin{figure}
  \centering
  \begin{subfigure}[b]{0.4\textwidth}
    \includegraphics{tikz-fig/poly-ex-123-ag-1.pdf}
  \end{subfigure}
  \begin{subfigure}[b]{0.4\textwidth}
    \includegraphics{tikz-fig/poly-ex-123-ag-2.pdf}
  \end{subfigure}\\
  \begin{subfigure}[b]{0.4\textwidth}
    \includegraphics{tikz-fig/poly-ex-123-ag-3.pdf}
  \end{subfigure}
  \begin{subfigure}[b]{0.4\textwidth}
    \includegraphics{tikz-fig/poly-ex-123-ag-4.pdf}
  \end{subfigure}
  \caption{Finding the first reachable point on \(\overline{P(3)P(4)}\). Resulting in \(t = 0.04\) marginally below point \(P(3)\). Note again that \(\delta(P(2), P(3)) > \varepsilon\). The point \(P(4)\) is not part of the simplification, it only lies on the line segment \(\overline{P(0)P(2)}\) by chance.}
  \label{fig:poly-ex-123-ag}
\end{figure}

From \(DP[1,2,3] = 0.04\) we can proceed to the end of the polyline to get a simplification of size \(2\) as seen in \cref{fig:poly-ex-265-ag}. A simplification of size \(1\) is not possible as that would be the line segment \(\overline{P(0)P(6)}\) but there is no point on that line segment within distance \(\varepsilon\) from \(P(3)\). 

\begin{figure}
  \centering
  \begin{subfigure}[b]{0.4\textwidth}
    \includegraphics{tikz-fig/poly-ex-265-ag-1.pdf}
  \end{subfigure}
  \begin{subfigure}[b]{0.4\textwidth}
    \includegraphics{tikz-fig/poly-ex-265-ag-2.pdf}
  \end{subfigure}\\
  \begin{subfigure}[b]{0.4\textwidth}
    \includegraphics{tikz-fig/poly-ex-265-ag-3.pdf}
  \end{subfigure}
  \caption{Finding the first reachable point on \(\overline{P(5)P(6)}\). With that we have found a simplification for the whole polyline.}
  \label{fig:poly-ex-265-ag}
\end{figure}

We can also convince ourselves that the solution must be correct as the application of the algorithm from \citeauthor{computing_the_frechet_distance_between_two_polygonal_curves} guarantees that the Fréchet distance is at most \(\varepsilon\) by construction and even yields the points that can be used to construct suitable functions that show this. 

For completeness sake the entries on the layer \(k = 2\) are listed in \cref{tab:exlayer2}. The entry \((k - 1, i', j')\) they reference from the previous layer is added after the tuple. This predecessor tuple is not necessarily unique, in fact, every single one of these has at least two different possibilities for \(i', j'\).
\begin{table}[ht]
\centering
\begin{tabular}{|ccc|}
\hline
$(2,4,3):(2, 1)$ & $(2,4,4):(2, 1)$ & $(2,5,4):(2,4)$ \\
$(2,5,5):(2,4)$ & $(2,6,5):(2,4)$ & \\
\hline
\end{tabular}
	\caption{Valid entries for layer \(k = 2\). Each entry proceeds from \((1, i', j')\) where \(i'\) and \(j'\) are annotated as a tuple after the colon.}
\label{tab:exlayer2}
\end{table}

In total, for all three layers there were only \(1 + 17 + 5 = 23\) entries, far smaller than \(7^4 = 2401\), the (approximate) number of iterations iterations over all \(i, j, i', j'\). This motivates ideas to reduce the runtime.

\subsection{Optimizations}
\label{ssec:optimizations}
Based on the just described algorithm we outline optimizations that greatly reduce the \(\Oh(k^*n^5)\) runtime in practice. The most effective optimizations circumvent the theoretical runtime by skipping iterations. 

We first note that for the entry \((k, i, j)\) in the dynamic program we do not always need to iterate over all possible \(i' < i\) and \(j' \leq j\). As we are interested in the minimal value \(t\) on the line segment \(\overline{v_{j}v_{j+1}}\) that can be reached, we can stop further search if we have reached a lower bound for that value. Such a lower bound is \(\hat t_0'(\overline{v_{j}v_{j+1}}, i)\), i.e. the modified solution to the \cref{eq:eq_solve_main}, which is tight as for a sufficiently large \(k\) this value must be reached eventually. This allows an early break out of the search through the \(i' < i, j'\leq j\) and a speed up of upto quadratic runtime. This is the \emph{local minimality} optimization. 

Another optimization based on a similar insight is the following: If the entry at \((k-1, i, j) = \hat t_0'(\overline{v_{j}v_{j+1}}, i)\), i.e., we have found a simplification for the same \(i\) and \(j\) that has already reached the optimal value, we know this cannot be improved in the current layer \(k\) or any further one. This means, the entry at \((k, i, j)\) can never be used by some entry \((k + 1, i'', j'')\) because of minimality as any such entry that uses \((k, i, j)\) could have already used \((k-1, i, j)\) for \((k, i'', j'')\). This enables us to completely ignore the computation of the value, resulting in a continuous speed up of the algorithm while it runs as once this happens for an entry, it will happen for all further ones with a higher value of \(k\) so with each layer more entry computations can be skipped. This is the \emph{global minimality} optimization.

Another way to skip computations is to ignore all \(i\) with \(i < k\) as well as all \(i'\) with \(i' < k - 1\) in the computations as there can never be a simplification of \(P[0\dots i]\) that uses more than \(i\) line segments thus it must always hold that \(i \geq k\). This optimization does not get a name and all our implementations have this inherently.

A last simple optimization of a similar type which is particularly useful for well-behaved polylines is to not even start the iterations over the \(i', j'\) if there is no solution \(\hat t_0'(\overline{v_{j}v_{j+1}}, i)\). This happens often if there are not too many line segments close to each other. This is the \emph{reachability} optimization. 



