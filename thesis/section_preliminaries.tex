\section{Preliminaries}
\label{sec:preliminaries}

This section introduces the notation, conventions, and definitions used throughout this report.

We denote the set of natural numbers by \(\N\), which includes \(0\), and define \(\N_+ = \N \setminus\set{0}\) as the set of positive natural numbers.

\subsection{Polylines}
\label{ssec:polylines}

The central geometric object of our study is the \emph{polyline}, defined as follows:

\begin{definition}[Polyline]
  Let \(d\in \N_+\) and \(n \in \N\) be natural numbers, and let \(u_0, \dots, u_n \in \R^d\) be points in \(d\)-dimensional space.

  The sequence \(P = \angl{u_0, \dots, u_n}\) is a \(d\)-dimensional \emph{polyline} of length \(n\). It consists of \(n+1\) points connected by \(n\) line segments.
  \begin{itemize}
    \item We interpret \(P\) as a continuous function \(P:[0,n] \to \R^d\) such that \(P(i) = u_i\) for \(i \in \set{0, \dots, n}\).
      Points in between integers are linearly interpolated: for \(t \in [0, 1]\) and \(i \in \set{0, \dots, n - 1}\), we set \(P(i + t) = (1- t)u_i + t u_{i+1} = u_i + t(u_{i+1} - u_i)\).
    \item We denote by \(P[t' \dots t]\) the subpolyline from parameter \(t' \in [0, n]\) to \(t \in [t', n]\). Formally,
			\[P[t' \dots t] = \angl{P(t'), P(\floor{t'} + 1),  P(\floor{t'} + 2), \dots, P(\ceil{t} - 1), P(t)}.\]
  \end{itemize}
\end{definition}

We denote the dimension by \(d\) throughout the report. Polylines are typically denoted by capital letters such as \(P\) and \(Q\). Single line segments are often denoted by \(e\).

The length of a polyline is typically denoted by \(n\), \(p\), or \(q\), where \(p\) and \(q\) are the lengths of \(P\) and \(Q\), respectively, and \(n\) is used when discussing a single polyline.

Following \citeauthor{polyline_simplification_has_cubic_complexity_bringmannetal}, we differentiate between \(\O\) and \(\Oh\) notation, where the latter hides polynomial factors in \(d\)\footnote{In this thesis, no exponential factors appear, so \(\Oh\) hides all factors depending on \(d\).}.

\subsection{Distances}
\label{ssec:distances}
We distinguish between distances between points and distances between polylines.

\begin{definition}[Distances]\label{def:point_distance}
  Let \(d \in \N_+\) and \(\ell \geq 1\).
  \begin{itemize}
    \item The \emph{unnormalized \(\ell\)-Minkowski distance} \(\delta'_\ell\) is defined as
      \[\delta'_\ell:\R^d \times \R^d \to \R_{\geq 0}, (u, v) \mapsto \sum_{i = 1}^d |u_i - v_i|^\ell.\]
    \item The \emph{(normalized) \(\ell\)-Minkowski distance} \(\delta_\ell\) is
      \[\delta_\ell:\R^d \times \R^d \to \R_{\geq 0}, (u, v) \mapsto \delta'_\ell(u, v)^{\frac1\ell} = \parenth{\sum_{i = 1}^d |u_i - v_i|^\ell}^{\frac1\ell}.\]
    \item The special case of \(\delta_2'\) is called the \emph{unnormalized Euclidean distance} and \(\delta_2\) the \emph{(normalized) Euclidean distance}.
    \item When \(\ell = 1\), the unnormalized and normalized versions coincide. We call \(\delta_1' = \delta_1\) the \emph{Manhattan distance}.
    \item We define the \emph{Chebyshev distance} \(\delta'_\infty = \delta_\infty\) as
      \[\delta_\infty:\R^d \times \R^d \to \R_{\geq 0}, (u, v) \mapsto \max_{i = 1, \dots, d} |u_i - v_i|.\]
    \item We define the auxiliary function \(\nu_\ell:\R_{\geq 0} \to \R_{\geq 0}\) as \(\nu_\ell(x) = x^\ell\) for \(\ell \neq \infty\) and \(\nu_\infty(x) = x\).
  \end{itemize}
  The subscript \(\ell\) is omitted when clear from context.
\end{definition}

The Euclidean distance (\(\ell = 2\)) is the most widely used metric. The Manhattan distance (\(\ell = 1\)) and Chebyshev distance (\(\ell = \infty\)) are computationally simpler, as they avoid roots.
Other Minkowski distances are less common due to numerical instability and lack of geometric interpretation. The unnormalized variants will later allow us to avoid explicit root computations in algorithms.

\begin{definition}[Fréchet Distance]\label{def:frechet}
  Let \(\delta\) be a normalized distance. The \emph{Fréchet distance} \(\delta^F\) between two polylines \(P\) and \(Q\) of lengths \(p\) and \(q\), respectively, is
	\[\delta^F(P, Q) = \inf_{\substack{f \in \mathcal{C}([0,1], [0, p]) \\ g \in \mathcal{C}([0,1], [0, q])}} \max_{t \in [0,1]}\delta(P(f(t)), Q(g(t))),\]
	where \(\mathcal{C}([a,b], [c,d])\) denotes the set of continuous, non-decreasing functions \(f\) mapping \([a,b]\) to \([c,d]\) with \(f(a) = c\) and \(f(b) = d\).
\end{definition}

\begin{definition}[Polyline Simplification]
	Given a polyline \(P\) of length \(n\), an error parameter \(\varepsilon > 0\), and a distance \(\delta\), the \emph{global Fréchet simplification problem} is to find a minimal subsequence \(Q\) of the vertices of \(P\) that includes the start point \(P(0)\) and the end point \(P(n)\), and satisfies \(\delta^F(P, Q) \leq \varepsilon\).
\end{definition}

This differs from \emph{local} simplification, where each line segment \(e = \overline{S(i)S(i+1)}\) of the simplification must satisfy \(\delta^F(e, P[j' \dots j]) \leq \varepsilon\) for its corresponding subpolyline. We focus on the global Fréchet case but cover the local one briefly in \cref{sec:polyline-simplification}.

We refer to a solution \(Q\) of the global Fréchet simplification problem as a \emph{polyline simplification}. Furthermore, any subsequence \(Q\) of \(P\) that includes \(P(0)\) and \(P(n)\) and satisfies \(\delta^F(P, Q) \leq \varepsilon\) will be referred to as a \emph{non-optimal simplification}; it meets all criteria except minimality.

\subsection{Properties of Distances}
All introduced distances are \emph{metrics} on \(\R^d\)~\cite{metric_spaces}:

\begin{definition}[Metric Spaces]\label{def:metric}
  Let \(X\) be a set and \(\delta:X\times X \to \R\). Then \(\delta\) is a \emph{metric} on \(X\) if for all \(a, b, c \in X\):
  \begin{itemize}
    \item \(\delta(a, b) \geq 0\) with equality if and only if \(a = b\), \hfill (Positivity)
    \item \(\delta(a, b) = \delta(b, a)\), and \hfill (Symmetry)
    \item \(\delta(a, c) \leq \delta(a, b) + \delta(b, c)\). \hfill (Triangle Inequality)
  \end{itemize}
  A set \(X\) together with a metric \(\delta\) is called a \emph{metric space}.
\end{definition}

\begin{observation}\label{obs:unnormalize}
  Let \(\ell \in [1, \infty]\), \(\varepsilon > 0\), and \(u, v \in \R^d\). Then
    \[\delta_\ell(u, v) \leq \varepsilon \iff \delta_\ell'(u, v) \leq \nu_\ell(\varepsilon).\]
\end{observation}

\begin{lemma}\label{lem:distance_properties}
	Let \(\delta\) be any Minkowski distance (including Chebyshev). For all \(u, v, w, x \in \R^d\), \(a \in \R\), and \(t \in [0, 1]\):
  \begin{enumerate}
		\item \(\delta(u, v) = \delta(u - w, v - w)\), \hfill (Translation Invariance)
		\item \(\delta(a u, a v) = |a| \delta(u, v)\), \hfill (Homogeneity)
		\item If \(\delta(u, w) \leq \varepsilon\) and \(\delta(v, w) \leq \varepsilon\), then \(\delta((1-t)u + tv, w) \leq \varepsilon\). \hfill (Convexity)
		\item \(\delta^F(\overline{uv}, \overline{wx}) \leq \varepsilon\) if and only if \(\delta(u, w) \leq \varepsilon\) and \(\delta(v, x) \leq \varepsilon\).
	\end{enumerate}
\end{lemma}

Property (3) implies that the set \(\set{u \mid \delta(u, v) \leq \varepsilon }\) is convex for fixed \(v \in \R^d\) and \(\varepsilon \geq 0\), motivating the name. Property (4) provides a simple characterization of the Fréchet distance between two line segments.

\begin{proof}
  \begin{enumerate}
    \item Follows directly from the definitions.
    \item Follows directly from the definitions.
		\item Assume \(\delta(u, w) \leq \varepsilon\) and \(\delta(v, w) \leq \varepsilon\). Let \(z = (1-t)u + tv\). Then,
			\begin{flalign*}
				\delta(z, w) &= \delta(z - w, 0) && \text{(Translation)} \\
        	&= \delta((1-t)(u-w) + t(v-w), 0) \\
					&\leq \delta((1-t)(u-w), 0) + \delta(t(v-w), 0) && \text{(Triangle Inequality)} \\
					&= (1-t)\delta(u-w, 0) + t\delta(v-w, 0) && \text{(Homogeneity)} \\
					&= (1-t)\delta(u, w) + t\delta(v, w) && \text{(Translation)} \\
					&\leq (1-t)\varepsilon + t\varepsilon = \varepsilon.
    \end{flalign*}
	\item (\(\Rightarrow\)) This direction follows because the reparameterizations \(f\) and \(g\) must satisfy \(f(0) = g(0) = 0\) and \(f(1) = g(1) = 1\), so the endpoints are matched at \(t=0\) and \(t=1\).

		(\(\Leftarrow\)) For the backward direction, consider the linear reparameterizations \(f(s) = s\) and \(g(s) = s\) for \(s \in [0,1]\). Then, for any \(s \in [0, 1]\),
    \begin{flalign*}
			\delta(P(f(s)), Q(g(s))) &= \delta((1-s)u + sv, (1-s)w + sx) \\
			&= \delta((1-s)(u-w) + s(v-x), 0) && \text{(Translation)} \\
			&\leq \delta((1-s)(u-w), 0) + \delta(s(v-x), 0) && \text{(Triangle Inequality)} \\
			&= (1-s)\delta(u, w) + s\delta(v, x) && \text{(Homogeneity and Translation)} \\
			&\leq (1-s)\varepsilon + s\varepsilon = \varepsilon.
    \end{flalign*}
		Since the maximum over \(s \in [0,1]\) is at most \(\varepsilon\), the Fréchet distance is at most \(\varepsilon\).
  \end{enumerate}
\end{proof}
