\section{Adapting Imai and Iri to the Global Fréchet Distance}
\label{sec:global_imai_iri}

In this section, we revisit the well-known \citeauthor{computational_geometric_methods_for_polygonal_approximations_of_a_curve}~\cite{computational_geometric_methods_for_polygonal_approximations_of_a_curve} algorithm for local Fréchet polyline simplification and demonstrate how to adapt it to the global Fréchet setting. In doing so, we rediscover the simplification algorithm from \citeauthor{global_curve_simplification}. This presents a new perspective on their algorithm and the link to the \citeauthor{computational_geometric_methods_for_polygonal_approximations_of_a_curve} algorithm might allow to adapt optimizations from the local setting.

Throughout this section we fix \(\varepsilon > 0\) and a distance \(\delta\). Further, let \(P\) be a polyline of length \(n\).

\subsection{Local Case}

We recall the algorithm in the local setting to identify where modifications are needed to adapt it to the global setting. 

The algorithm consists of two main parts: first, a shortcut graph is constructed and then, a shortest path computation yields the simplification.

The shortcut graph is a directed acyclic graph whose vertices correspond to the polyline vertices. A directed edge \((P(i), P(j))\) with \(i < j\) exists if \(\delta^F(P[i \dots j], \overline{P(i)P(j)}) \leq \varepsilon\) (i.e., \(\overline{P(i)P(j)}\) is a shortcut for the subpolyline \(P[i \dots j]\)). 

To determine whether \(\delta^F(P[i \dots j], \overline{P(i)P(j)}) \leq \varepsilon\), we use the algorithm from \citeauthor{computing_the_frechet_distance_between_two_polygonal_curves}, modified versions of which are outlined in \cref{ssec:alt_godau,ssec:implicit_frechet_decision}. Here, the simpler classical version suffices. This test requires \(\Oh(n)\) time and as there are \(\O(n^2)\) many pairs to test, constructing the shortcut graph requires \(\Oh(n^3)\) time.

Each path from \(P(0)\) to \(P(n)\) corresponds to a valid, not necessarily optimal, local simplification of the polyline by construction and it is obvious that all such simplifications are captured in the shortcut graph. Thus, computing the shorted \(P(0)-P(n)\) path in this graph yields an optimal local simplification. This can be done easily in \(\O(n^2)\).

The total runtime is determined by the cubic graph construction phase thus the total runtime is \(\Oh(n^3)\).

\subsection{The Global Shortcut Graph}
In the global case the structure of these shortcuts is more complicated, necessitating a more sophisticated approach. To confront this problem, we store for each shortcut the set of all subpolylines for which it is a valid shortcut. 

\begin{definition}
  For a shortcut \(e = \overline{P(i')P(i)}\) we say that \((t', t) \in [0, n]^2\) with \(t' \leq t\) is \emph{\(e\)-admissible} if and only if \(\delta^F(P[t' \dots t], e) \leq \varepsilon\). We denote \(\mathcal{A}_e = \set{(t', t) \mid (t', t) \textrm{ is } e\textrm{-admissible}}\) as the set of \(e\)-admissible subpolylines.
\end{definition}

The following observation might seem complicated, but it follows directly from the definitions of admissibility and Fréchet distance. It captures our intuition that global simplifications are constructed of only admissible subpolylines.

\begin{observation}
	Let \(Q=\angl{P(u_0), \dots, P(u_q)}\) be a (not nessarily optimal) global simplification of \(P\) which means there are parameterizations increasing, continuous \(f, g\) with \(f([0, 1]) = [0, n]\) and \(g([0,1]) = [0, q]\) with \(\delta(P(f(\alpha)), Q(g(\alpha))) \leq \varepsilon\) for all \(\alpha \in [0, 1]\).

	Then, for \(i \in \set{1, \dots, q}\) and \(x \leq y \in [0, 1]\) such that \(g(x) = i - 1\) and \(g(y) = i\) it holds that 
	\[\delta^F(P[f(x) \dots f(y)], \overline{P(g(x)), P(g(y))}) \leq \varepsilon.\]
	This means \((f(x), f(y))\) is \(e\)-admissible for \(e = \overline{P(g(x), P(g(y)))} = \overline{Q(i-1), Q(i)}\).
\end{observation}

Next, we explore the set of admissible subpolylines \(\mathcal{A}_e\) with the goal of establishing an efficient representation.

\begin{lemma}\label{lem:admissible_are_intervals}
	Let \(e = \overline{P(i')P(i)}\).
  \begin{enumerate}
		\item Let \((r, t)\in \mathcal{A}_e\) and \(r \leq t' \leq t\) with \(\delta(P(t'), P(i')) \leq \varepsilon\). Then \((t', t) \in \mathcal{A}_e\).
		\item Let \((t', r)\in \mathcal{A}_e\) and \(t' \leq t \leq r\) with \(\delta(P(t), P(i)) \leq \varepsilon\). Then \((t', t) \in \mathcal{A}_e\).
  \end{enumerate}
\end{lemma}

\begin{proof}
	We show the first statement. The second one is analogous. We create suitable parameterizations \(f\) and \(g\) to show \(\delta^F(P[t' \dots t], e) \leq \varepsilon\). As \((r,t)\) is \(e\)-admissible, there are functions increasing, bijective parameterizations \(f':[0,1] \to [r \dots t]\) and \(g':[0,1] \to [0,1]\) (here we shift the range of \(f'\) to be consistent with \(P\) and not the subpolyline \(P[r \dots t]\)).

	Since \(f'\) is surjective onto \([r, t]\) and \(t' \in [r, t]\), there is \(x \in [0, 1]\) with \(f'(x) = t'\). This means that \(\delta(P(t'), e(g'(x))) \leq \varepsilon\). By the convexity property in \cref{lem:distance_properties}, \(\delta(P(t'), e(\lambda)) \leq \varepsilon\) for all \(\lambda \in [0, g'(x)]\). With this, we construct the parameterizations \(f\) and \(g\) such that \(f\) is the zero function until \(g\) assumes the value \(g'(x)\). After this point both functions are identical to \(f'\) and \(g'\) respectively (up to a shift in the range).
\end{proof}

\begin{definition}
	For sets \(A\) and \(B\) we define \(A \times_{\leq} B\) to be 
		\[A \times_{\leq} B = \set{(a, b) \in A \times B \mid a \leq b}.\]
\end{definition}

Using \cref{lem:admissible_are_intervals}, we define an efficient representation of \(\mathcal{A}_e\).

\begin{lemma}\label{lem:admissible-rep-1}
	Let \(e = \overline{P(i')P(i)}\) be a line segment. Denote \(\mathcal{I} = \set{t \in [0,n] \mid \delta(P(i'), P(t)) \leq \varepsilon}\) and \(\mathcal{J} = \set{t \in [0,n] \mid \delta(P(i), P(t)) \leq \varepsilon}\). There are closed intervals \(I \subseteq [\min \mathcal{I}, \max \mathcal{I}]\) and \(J \subseteq [\min \mathcal{J}, \max \mathcal{J}]\) such that 
	\[\mathcal{A}_e = \parenth{\mathcal{I} \cap I} \times_\leq \parenth{\mathcal{J} \cap J}.\]
\end{lemma}

\begin{proof}
	The lemma is trivial if \(\mathcal{A}_e\) is empty. Suppose it is not empty. 
	Let \(m_I \in [0, n]\) be minimal such that there is \(n \geq m_I\) with \((m_I, n)\) is \(e\)-admissible and similarly let \(M_I\) be maximal such that a suitable \(n\) exists. Furthermore, let \(m_J \in [0,n]\) be minimal such that \((m_I, m_J)\) is \(e\)-admissible and \(M_J\) be maximial such that \((m_I, M_J)\) is \(e\)-admissible. 

	We set \(I = [m_I, M_I]\) and \(J = [m_J, M_J]\) and show both inclusions separately.

	(\(\subseteq\)) Let \((t', t)\) be \(e\)-admissible. Then \(t' \in \mathcal{I}\), \(t \in \mathcal{J}\), and \(t' \leq t\) by definition. By definition of \(m_I, M_I, m_J, M_J\) and using \cref{lem:admissible_are_intervals} it follows that \(t' \in I\) and \(t \in J\).

	(\(\supseteq\)) Let \((t', t)\) satisfy \(t' \leq t\), \(m_I \leq t' \leq M_I\), \(t' \in \mathcal{I}\), \(m_J \leq t \leq M_J\), and \(t \in \mathcal{J}\). We repeatedly utilize \cref{lem:admissible_are_intervals} in the following. As \(m_I \leq t' \leq M_J\), \(t' \in \mathcal{I}\), and \((m_I, M_J)\) is \(e\)-admissible, \((t', M_J)\) is \(e\)-admissible. Similarly, as \(t' \leq t \leq M_J\), \(t \in \mathcal{J}\), and \((t', M_J)\) is \(e\)-admissible, so must \((t', t)\) be.
\end{proof}


\begin{definition}[Global Shortcut Graph]
	For a polyline \(P = \angl{v_0, \dots, v_n}\) and \(\varepsilon > 0\) we define the global shortcut graph as the directed acyclic graph \(G = (V, E)\) with \(V = \set{v_0, \dots, v_n}\) and \(E = \set{(v_i, v_j) \mid i < j, e = \overline{v_iv_j}, \mathcal{A}_e \neq \emptyset}\).

	We define a labelling function on the edges 
	\[S(e) = (I = [m_I, M_I], J = [m_J, M_J]),\]
	where \(m_i, M_I, m_J, M_J\) are taken from the proof of \cref{lem:admissible-rep-1}.
\end{definition}

Let us briefly analyze the space consumption of the global shortcut graph: for each line segment, we only need to store 4 interval boundaries, resulting in total \(\O(n^2)\) space consumption for the edges. The vertices require only linear space, so in total this requires \(\O(n^2)\) space. 

However, we did not consider the solution intervals \(\mathcal{I}\) and \(\mathcal{J}\) (see \cref{lem:admissible_are_intervals}). For each line segment there are in the worst case linearly many such intervals which would require in total cubic space. However, these intervals are only dependent on the points, not the line segments themselves, meaning there are linearly many closed intervals per polyline vertex and thus in total quadratically many. Thus, the total information of the global shortcut graph requires \(\O(n^2)\) space, with potentially less space if the graph is sparse (many \(\mathcal{A}_e\) are empty) and each point has only few such intervals. 






