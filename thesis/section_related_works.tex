\section{Related Work}
\label{sec:related_work}

In this section, we cover previous work on the topic of polyline simplification and how our work extends it. We focus here on works that apply to the thesis as a whole. More specific references relevant to individual areas are discussed in the subsequent sections.

Polyline simplification has been studied extensively due to its many applications. One of the oldest algorithms is the Douglas-Peucker algorithm \cite{algorithms_reduction_number_points_caricature}, proposed over fifty years ago, which is fast to compute but provides no optimality guarantees. To formalize an optimization objective, various distance measures between polylines have been studied, with the Hausdorff and Fréchet distances being the most widely used; both can be applied in numerous ways.

\citeauthor{computing_the_frechet_distance_between_two_polygonal_curves}~\cite{computing_the_frechet_distance_between_two_polygonal_curves} showed how to compute the Fréchet distance between two polylines and how to solve the corresponding decision problem. They also stated the problem of polyline simplification, which they called curve approximation, but did not provide a solution. The version they considered was minimization under the global Fréchet distance. Thus, this formulation of polyline simplification is at least 30 years old but has received relatively little attention.

The related local version, on the other hand, has been discussed in more detail. One of the most well-known locally optimal algorithms is by Imai and Iri \cite{computational_geometric_methods_for_polygonal_approximations_of_a_curve}, which has been the subject of numerous subsequent studies and improvements for various use cases.

The global version has only recently been studied in depth. \citeauthor{on_optimal_polyline_simplification_using_the_hausdorff_and_frechet_distance}~\cite{on_optimal_polyline_simplification_using_the_hausdorff_and_frechet_distance} showed that simplification under the global Fréchet distance can be solved in polynomial time, as well as showing that simplification under the global Hausdorff distance is NP-hard. However, their algorithm has a quintic runtime in the best case and a sextic runtime in the worst case. Although polynomial, this complexity makes it infeasible in practice, and it has, to our knowledge, not been implemented previously. In \cref{sec:algorithm_implementation}, we discuss this algorithm along with optimizations, practical considerations, and parallelization techniques that make it usable in practice.

\citeauthor{polyline_simplification_has_cubic_complexity_bringmannetal}~\cite{polyline_simplification_has_cubic_complexity_bringmannetal} improved upon the algorithm of \citeauthor{on_optimal_polyline_simplification_using_the_hausdorff_and_frechet_distance}, achieving cubic runtime. They further provided a conditional lower bound based on the \(\forall\forall\exists\)-Orthogonal Vector Hypothesis, ruling out subcubic runtime for various cases in high dimensions. For this algorithm, we also provide detailed explanations and practical optimizations to improve its performance in \cref{sec:cubic_algo}.

