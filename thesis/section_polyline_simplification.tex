\section{Polyline Simplification and Basic Properties}\label{sec:polyline-simplification}
In this section, we discuss and illustrate different types of simplification objectives that have been studied. 
This yields valuable insights for the rest of this work as well as highlights differences between most of the previously studied
types and the global (vertex-restricted) simplification under the Fréchet distance that we mainly focus on.

We mainly use the classification of \citeauthor{global_curve_simplification}~\cite{global_curve_simplification} as a basis. 

There are two aspects in which the types of simplification differ from each other:
\begin{enumerate}
  \item The set from which to choose the points of the simplification and 
	\item The distance function between polylines that is used to measure the quality of the simplification. 
\end{enumerate}

These two aspects are not completely independent of each other.

In \cref{sec:preliminaries} we have already defined one distance function between polylines which is the \emph{Fréchet distance}. The other commonly used function is the \emph{Hausdorff distance}

\begin{definition}[Hausdorff Distance]
  Let \(P\) and \(Q\) be polylines of length \(p\) and \(q\) respectively in \(d\)-dimensions. Let \(\delta\) be a distance function on points. 

	\begin{itemize}
		\item The \emph{directed Hausdorff distance} \(\delta^{dH}(P, Q)\) from \(P\) to \(Q\) is defined as 
		\[\delta^{dH}(P, Q) = \max_{s \in [0, p]}\min_{t \in [0, q]} \delta(P(s), Q(t))\].
		\item The \emph{(undirected) Hausdorff distance} \(\delta^{H}(P, Q)\) of \(P\) and \(Q\) is defined as 
		\[\delta^{H}(P, Q) = \max(\delta^{dH}(P, Q), \delta^{dH}(Q, P))\].
	\end{itemize}

	Note that the directed Hausdorff distance is not symmetric while the undirected Hausdorff distance is.
\end{definition}

Both the directed and undirected Hausdorff distance as well as the Fréchet distance can be used with different types of point distances, with the most commonly used being the Euclidean distance, the Manhattan distance, and the Chebyshev distance.

These distances are used to measure the simplification against the original polyline. In the case of the directed Hausdorff distance there are two different ways: From the simplification to the polyline or the other way around. 
Further, for both the Hausdorff and the Fréchet distance, a common alternative way is to compare the simplification only \emph{locally}.

To understand local simplifications, we need to define which points are allowed to be used in the simplification. \citeauthor{global_curve_simplification} distinguish three types:
\begin{itemize}
  \item \emph{Vertex-restricted}: Only the finitely many points that define the polyline are allowed, i.e., for a polyline \(P = \angl{u_0, \dots, u_n}\) a simplification is only allowed to contain a subsequence of \(u_0, \dots, u_n\) without changing the order.
	\item \emph{Curve-restricted}: Only points that lie on the polyline are allowed, i.e., for a polyline \(P\) of length \(p\) any point \(P(t)\) for \(t \in [0, p]\) is allowed. This sequence of \(t\) used must be increasing.
	\item \emph{Non-restricted}: Any point in the ambient space is valid 
\end{itemize}

Local simplifications is mainly used in the vertex-restricted setting but also works for the curve-restricted one. There we apply the distance function not between the simplification \(Q\) and the original \(P\) but instead apply it between the line segment of \(Q\) and the respective subpolyline in \(P\) and take the maximum over all of these. More specifically, for a polyline distance function \(\delta^P\) we measure the distance the local distance \(\delta^{LP}\) as 
\[\delta^{LP}(P, Q) = \max_{i = 1, \dots, q} \delta^P(P[k_{i-1}\dots k_i], Q[i-1 \dots i]),\]
where \(P = \angl{u_0, \dots, u_p}\) and \(Q = \angl{u_{k_0}, \dots, u_{k_q}}\).

In contrast to the local setting, there is also the \emph{global} one which we study for the Fréchet distance in this work. In this setting the non-restricted cases can also be studied.

In the following, we discuss the vertex-restricted local and global (undirected) Hausdorff and Fréchet setting. For more information on other types of global setting we refer to \citeauthor{global_curve_simplification}'s paper.

\subsection{Hausdorff Distance}
In this thesis, we study only the Fréchet distance, not the Hausdorff distance. The global Hausdorff distance is among the four variants we discuss here the least useful one. \citeauthor{on_optimal_polyline_simplification_using_the_hausdorff_and_frechet_distance} showed that it is NP-hard to find such simplifications~\cite{on_optimal_polyline_simplification_using_the_hausdorff_and_frechet_distance} which is less than ideal when it comes to practically applicable algorithms. Further, it does not capture the contour of a polyline as the order of points is not involved in the measurement. This causes oversimplifications. See \cref{fig:polyline-ex-hausdorff} to see an example why using the Hausdorff distance may be problematic to measure similarity.

\begin{figure}[b]
  \centering
  \includegraphics{./tikz-fig/polyline-ex-hausdorff.pdf}
  \caption{Two polylines with a comparatively high Fréchet distance but small Hausdorff distance. Note that each point on any of the two polylines is close to some point of the other polylines which causes a small Hausdorff distance but the completely different contours cause a high Fréchet distance.}
  \label{fig:polyline-ex-hausdorff}
\end{figure}

The local Hausdorff does not have the same problems as applying the Hausdorff distance locally guarantees a coarse global order on the simplification. Further, whereas the global Hausdorff variant is the hardest to solve, the local Hausdorff setting seems to be the simplest. Currently, it  is the only variant for which subcubic algorithms are known in dimensions \(d \geq 3\) (see, e.g., \cite{efficiently_approximating_higher_dim}). 

The local Hausdorff variant has already been covered extensively which is why we do not expand further on it. 

\subsection{Local Fréchet}
The local Fréchet variant enforces locality constraints just like the local Hausdorff variant but using the Fréchet distance. Unlike the Hausdorff distance, the Fréchet distance already enforces the correct point ordering so the local constraints are technically unnecessary. The reason why the local version is more studied than the global version is that the locality constraints simplify the problem. In fact, the definition of the local variants gives rise to an algorithm, the algorithm from \citeauthor{computational_geometric_methods_for_polygonal_approximations_of_a_curve}~\cite{computational_geometric_methods_for_polygonal_approximations_of_a_curve}: First, for each pair of points we test if the line segment between is a valid shortcut, i.e., the Fréchet distance between the line segment and the respective subpolyline is within the error parameter \(\varepsilon\). Using this information we construct the shortcut graph where the vertices are the points of the polyline and the edges are the shortcuts. Finally a shortest path from the first to the last point is found.

This algorithm can also be used for the local Fréchet distance. For a full algorithm we need to be able to test if the Fréchet distance between two polylines is within a certain bound. This can be done with the algorithm from \citeauthor{computing_the_frechet_distance_between_two_polygonal_curves}~\cite{computing_the_frechet_distance_between_two_polygonal_curves} which we discuss in \cref{sec:algorithm_implementation}. This can generally be done in \(\O(pq\) for two polylines of length \(p\) and \(q\) respectively and as one of the polylines is always a line segment (the shortcut) it can be solved in linear runtime. The complete algorithm can be implemented cubic runtime.

The local Fréchet variant has seen improvements to this runtime in small dimensions. \citeauthor{polyline_simplification_under_the_local_frechet_distance_has_almost_quadratic_runtime_in_2d_storandtetal}~\cite{polyline_simplification_under_the_local_frechet_distance_has_almost_quadratic_runtime_in_2d_storandtetal} present an algorithm to solve the problem in \(\O(n^2\log n)\) with quadratic runtime when using the Manhattan, Chebyshev distance, or when the polyline satisfies some well-behavedness property. Their algorithm works in two dimensions and is itself a refinement of the algorithm from \citeauthor{computing_the_frechet_distance_between_two_polygonal_curves}.

\subsection{Global Fréchet}
Unlike its local counterpart, the global Fréchet variant has seen less research. As far as we are aware only three papers have actually researched the topic.

First, \citeauthor{on_optimal_polyline_simplification_using_the_hausdorff_and_frechet_distance}~\cite{on_optimal_polyline_simplification_using_the_hausdorff_and_frechet_distance} showed that the global Hausdorff variant is NP-hard and developed a first polynomial algorithm for the global Fréchet variant.

\citeauthor{polyline_simplification_has_cubic_complexity_bringmannetal}~\cite{polyline_simplification_has_cubic_complexity_bringmannetal} improved this algorithm to achieve cubic runtime and give matching conditional lower bounds which rule out subcubic algorithms in higher dimensions for both of the local and the global Fréchet variant. 

Finally, \citeauthor{global_curve_simplification}~\cite{global_curve_simplification} give a formal classification of global variants and present multiple results. For the global Fréchet variant that we discuss here they present an additional cubic algorithm.

\begin{figure}[b]
  \centering
  \includegraphics{./tikz-fig/polyline-ex-local-global-f.pdf}
  \caption{This polyline has a global Fréchet simplification consisting of two line segments but no local simplification of the same size.}
  \label{fig:polyline-ex-local-global-f}
\end{figure}

\subsection{Basic Properties}
We now investigate some elementary properties about polylines and their simplifications. We use many of these implicitly in \cref{sec:evaluation} to standardize the experiments as well as for automated testing in our implementations.

\begin{lemma}[Monotonicity of Minkowski Distances]\label{lem:monotonicity_minkowski}
  Let \(d \in \N_+\) be a dimension and \(1 \leq k \leq \ell \leq \infty\).
	\begin{enumerate}
		\item Let \(u, v \in \R^d\) be points. Then \(\delta_\ell(u,v) \leq \delta_k(u, v)\).
		\item Let \(P\) and \(Q\) be \(d\) dimensional polylines. Then \(\delta_\ell^F(P, Q) \leq \delta_k^F(P, Q)\).
	\end{enumerate}
\end{lemma}

\begin{proof}
  \begin{enumerate}
		\item It suffices to show the inequalities for the underlying norms, i.e., show the following 
			\[\left(\sum_{i=1}^d |x_i|^\ell\right)^{1/\ell} \leq \left(\sum_{i=1}^d |x_i|^k\right)^{1/k}\]
			and 
			\[\max_{i=1, \dots, d} |x_i| \leq \left(\sum_{i=1}^d |x_i|^k\right)^{1/k}\]
			for an arbitrary \(x \in \R^d\).

			For the first inequality fix \(m \coloneq \max_{1,\dots, d}|x_i|\) and also note that 
			\[m = \left(\max_{i=1, \dots, d} |x_i|^{k}\right)^{1/k} \leq \left(\sum_{i=1}^d |x_i|^{k}\right)^{1/k}\]
			as each term is positive and the maximum component is included in the rightmost sum. This already shows the second inequality.

			For the first one we get:
			\begin{align*}
				\sum_{i=1}^d |x_i|^\ell &= \sum_{i=1}^d |x_i|^k|x_i|^{\ell - k} \\
				 &\leq \sum_{i=1}^d |x_i|^k m^{\ell - k} \\
				 &\leq \sum_{i=1}^d |x_i|^k \left(\sum_{i=1}^d |x_i|^{k}\right)^{(\ell-k)/k} \\
				 &\leq \left(\sum_{i=1}^d |x_i|^{k}\right)^{\ell/k} \\
			\end{align*}
			
			Raising both sides of this inequality by \(1/\ell\) yields the wanted inequality.

		\item The respective result for the Fréchet distance follows directly from the definition of the Fréchet distance and the just showed property for points as for all suitable functions \(f\) and \(g\) we get \(\delta_\ell(P(f(t)), Q(g(t))) \leq \delta_k(P(f(t)), Q(g(t)))\).
  \end{enumerate}
\end{proof}

\begin{corollary}[Simplification Size Monotonicity]\label{cor:size_monotonicity}
	Let \(d \in \N_+\) be a dimension, \(1 \leq k \leq \ell \leq \infty\),
  \(P\) be a \(d\)-dimensional polyline, and \(\varepsilon \geq 0\). Denote the size of the \(\varepsilon\)-simplification of \(P\) using \(\delta_k\) or \(\delta_\ell\) as \(S_k\) or \(S_\ell\) respectively. Then \(S_\ell \leq S_k\).
\end{corollary}

\cref{cor:size_monotonicity} is a simple sanity check that is fast to implement and use for automated testing. 

\begin{proof}
	We use \cref{lem:monotonicity_minkowski} with the polyline \(P\) and an arbitrary (not necessarily optimal) simplification \(Q\) of \(P\). As \(\delta_\ell^F(P, Q) \leq \delta_k^F(P, Q)\) for each such simplification \(Q\), it must hold that if it is a valid \(\varepsilon\)-simplification of \(P\) under \(\delta_\ell\) then it must also be one under \(\delta_k\). As we minimize the point count during simplification we get the wanted inequality.
\end{proof}

\begin{lemma}[Translation Invariance]\label{lem:translation_invariant}
	Let \(P\) be a \(d\)-dimensional polyline and \(x \in \R^d\). Let \(\varepsilon \geq 0\). If \(Q\) is an \(\varepsilon\)-simplification of \(P\) then \(Q-x\) is an \(\varepsilon\)-simplification of \(P-x\) where the subtraction is interpreted as a shift of each point of the polyline, i.e., \(P-x = \angl{u_0 - x, \dots, u_p - x}\) for \(P = \angl{u_0,\dots, u_p}\).
\end{lemma}

\begin{proof}
	All distances that we consider are invariant under translation by \cref{lem:distance_properties} which directly extends to polylines and thus the Fréchet distance does not change under simultaneous translation of both polylines by the same shift. Thus the simplifications also do not change (except for the translation itself). 
\end{proof}

Because of \cref{lem:translation_invariant} we fix the first point of each polyline to be the origin. The indices of the points in the simplification will remain the same before and after translation. 

\begin{corollary}[Rotation Invariances]\label{cor:rot_inv}
  When using the Euclidean distance \(\delta_2\), applying a rotation on the polyline only rotates the simplifications but does not alter them in any other way.

	For all other distances the same holds only for discrete rotations by \(90^\circ, 180^\circ\), or \(270^\circ\), i.e., swapping entries or negating entries.
\end{corollary}

\begin{proof}
  This is another property that simply follows from the underlying property of the distance function as under these rotations all distances are preserved. 
\end{proof}

\begin{lemma}[Polyline Scaling]\label{lem:scaling}
	Let \(P\) be a \(d\)-dimensional polyline and \(c > 0\). Let \(\varepsilon \geq 0\). If \(Q\) is an \(\varepsilon\)-simplification of \(P\) then \(cQ\) is an \(c\varepsilon\)-simplification of \(cP\) where the multiplication is a scalar multiplication on  each point of the polyline, i.e., \(cP = \angl{cu_0, \dots, cu_p}\) for \(P = \angl{u_0,\dots, u_p}\).
\end{lemma}

\cref{lem:scaling} allows us to cap the maximum length of line segments through the use of scaling the polyline and the \(\varepsilon\) which we use for data generation in \cref{sec:evaluation}\footnote{In \cref{sec:evaluation} we also impose a lower bound on the line segment lengths which this lemma does not justify. }.

\begin{proof}
	Fix an \(\varepsilon\)-simplification \(Q\) of \(P\) and suitable functions \(f\) and \(g\). Scaling all points also scales all distances appropriately as \(\delta((cP)(f(t)), (cQ)(g(t))) = \delta(cP(f(t)), cQ(g(t))) = c\delta(P(f(t)), Q(g(t)))\).
\end{proof}

\begin{remark}
	\cref{cor:size_monotonicity}, \cref{lem:translation_invariant}, \cref{cor:rot_inv}, and \cref{lem:scaling} also apply to the curve-restricted and non-restricted setting.
\end{remark}


\subsection{A First Simplfication Algorithm}
We conclude this section with a simple simplification algorithm for both the local and global Fréchet variant under any distance. This however only covers simplifications for \(\varepsilon = 0\).

We already know that \(\delta(u, v) = 0\) if and only if \(u = v\) for two points \(u\) and \(v\). The same is almost true for polylines. 

\begin{lemma}\label{lem:polyline_dist0}
	Let \(P\) and \(Q\) be polylines of length \(p\) and \(q\) respectively. Then \(\delta^F(P, Q) = 0\) if and only if 
	There are \(k \in \N\) with \(k \leq \min(p, q)\) and strictly monotonously increasing \(f:\set{0,\dots,k} \to \set{0,\dots,p}\) and \(g:\set{0,\dots,k} \to \set{0,\dots,q}\) with \(f(0) = g(0) = 0\), \(f(k) = p\), and \(g(k) = q\) such that 
	\begin{enumerate}
		\item \(P(f(t)) = Q(g(t))\) for all \(t \in \set{0, \dots, k}\).
		\item There are \(0 \leq t_{f(\ell-1)+1} \leq \cdots \leq t_{f(\ell)-1} \leq 1\) for all \(\ell \in \set{1, \dots, k}\) such that \newline \(P(i) = (1-t_i)P(f(\ell - 1)) + t_iP(f(\ell))\).
		\item There are \(0 \leq t_{g(\ell-1)+1} \leq \cdots \leq t_{g(\ell)-1} \leq 1\) for all \(\ell \in \set{1, \dots, k}\) such that \newline \(Q(i) = (1-t_i)Q(g(\ell - 1)) + t_iQ(g(\ell))\).
	\end{enumerate}
\end{lemma}

\begin{proof}
	First assume \(\delta^F(P, Q) = 0\) this means \(P(f'(t)) = Q(g'(t))\) for suitable functions \(f':\mathcal{C}([0,1], [0,p])\) and \(g':\mathcal{C}([0,1], [0,q])\) and all \(t \in [0,1]\). Set \(k \coloneq |\set{u \mid u = P(i) = Q(j), i = 0, \dots, p, j = 0, \dots, q}|\) and define \(f\) and \(g\) to map onto these \(u\) accordingly in order of \(i\) and \(j\). This satisfies the first property.

	For the other two properties we need to show that between all points that are consecutive via \(f\) or \(g\) only a line segment lies and not multiple bends. Further we need to show that the points on the line segments are in appropriate order. The second of which is guaranteed by the Fréchet distance. For the first property assume that there are at least two proper line segments. Then there must lie a point that lies on \(P\) on it. As the Fréchet distance is \(0\) the respective point must also be on \(Q\) but as it is between two line segments that do not form a single line segment it must be a proper point on \(Q\) and thus be part of \(f\) which is a contradiction to the definition of \(f\). We argue similarly for the third property.

	For the other direction, notice that these properties imply that the polylines are the same upto possible additions of points on line segments (without going back on the line segment). From this it is trivial to construct suitable functions \(f'\) and \(g'\) such that \(P(f'(t)) = Q(g'(t))\) for each \(t \in [0, 1]\).
\end{proof}

According to \cref{lem:polyline_dist0}, two polylines have distance \(0\) if and only if the only points by which they differ are points that lie on a line segment of the polyline. In other words, by removing all collinear points the polylines are the same.

Using this, it is trivial to construct a linear time algorithm to obtain optimal simplifications when \(\varepsilon = 0\). The start and end point must always be included and for all points inbetween we test if the point lies on the line segment defined by its two neighbors. If this is not the case we apped the point to the simplification and otherwise we skip it. 

\begin{algorithm}[ht]
  \DontPrintSemicolon
  \KwData{Polyline \(P\) of length \(n\)}
  \KwResult{Smallest \(0\)-simplification of \(P\)}
  \BlankLine
	\(simplification \gets (0)\)\;
	\For{\(i=1,\dots, n - 1\)}{
		\If{\(P(i)\) does not lie on \(\overline{P(i-1)P(i+1)}\)}{
			\(simplification.append(i)\)
		}
	}
	\(simplification.append(n)\)
  \caption{PolylineSimplificationWithEpsilon0(\(P\))}
  \label{algo:simplify_epsilon0}
\end{algorithm}

To test if a point lies on a line segment is trivial but for completeness sake we derive it. A point \(p\) lies on \(\overline{uv}\) if and only if there is \(t \in [0,1]\) such that \(p = u + t(v-u)\) or equivalently \(p - u = t(v-u)\). So we need \(t = \frac{p_i - u_i}{v_i - u_i}\) for all entries \(i\). To avoid divions by \(0\) we can reformulate it as 
\((p_{i-1} - u_{i-1})(v_i - u_i) = (p_i - u_i)(v_{i-1} - u_{i-1})\) for all \(i \geq 1\). To enforce the condition that \(t \in [0, 1]\) we require \(0 \leq \frac{p_i-u_i}{v_i - u_i} \leq 1\) which is equivalent to \(u_i \leq p_i \leq v_i \) if \(v_i \geq u_i\) and otherwise \(v_i \leq p_i \leq u_i\) for any component that does not contain the same value for \(u_i, v_i\) and \(p_i\). 

















